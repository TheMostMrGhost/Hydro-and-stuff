\documentclass[11pt,oneside]{book}
\usepackage[margin=1.2in]{geometry}
\usepackage[toc,page]{appendix}
\usepackage{graphicx}
\usepackage{natbib}
\usepackage{twoopt} % Do działania komend z kilkoma argumentami
\usepackage{cancel}
\usepackage{lipsum}
\usepackage{caption}
\usepackage[T1]{fontenc}
\usepackage{titlesec, blindtext, color}
\usepackage{xcolor,tikz}
\usepackage{tikz-cd}
\usepackage{caption}
\usepackage{subcaption}
\usetikzlibrary{patterns}
\usetikzlibrary{decorations.markings}
\usetikzlibrary{decorations.pathmorphing}
% \usepackage{amsmath,amssymb,amsthm,mathrsfs,amsfonts,xfrac,pifont,bbold,physics}
\usepackage[utf8]{inputenc}
\usepackage{amsthm}
\usepackage{pgfplots}
\usepackage[breakable, theorems, skins]{tcolorbox}
\usepackage[colorlinks = true,
            linkcolor = red,
            urlcolor  = blue,
            citecolor = red,
            anchorcolor = red]{hyperref}
\usepackage{enumitem}
\usepackage{TemplateMatma}
\usepackage{TemplateNotatkiPE}

\tikzset{
    partial ellipse/.style args={#1:#2:#3}{
        insert path={+ (#1:#3) arc (#1:#2:#3)}
    }
}


\renewcommand{\vec}[1]{\underline{#1}}

% -------------------------------------------------------------------
% Theorem Styles
% -------------------------------------------------------------------

\theoremstyle{definition} % Define theorem styles here based on the definition style (used for definitions and examples)
\newtheorem*{definition}{Definition}

\theoremstyle{plain} % Define theorem styles here based on the plain style (used for theorems, lemmas, propositions)
\newtheorem{theorem}{Theorem}[section]
\newtheorem{axiom}{Axiom}
\newtheorem{corollary}[theorem]{Corollary}
\newtheorem{lemma}[theorem]{Lemma}
\newtheorem{proposition}[theorem]{Proposition}
\newtheorem{postulate}{Postulate}

\theoremstyle{remark} % Define theorem styles here based on the remark style (used for remarks and notes)
\newtheorem*{solution}{Solution}


\newtheoremstyle{underline}% name
{}        % Space above, empty = `usual value'
{}              % Space below
{}              % Body font
{}    % Indent amount (empty = no indent, \parindent = para indent)
{}              % Thm head font
{.}             % Punctuation after thm head
{1.5mm}         % Space after thm head: \newline = linebreak
{{\underline{\textit{\thmname{#1}\thmnumber{ #2}}~\thmnote{(#3)}\unskip}}}  % Thm head spec

\theoremstyle{underline}

\newtheorem{remark}[theorem]{Remark}
\newtheorem{example}[theorem]{Example}
\newtheorem{claim}[theorem]{Claim}
\newtheorem{exercise}[theorem]{Exercise}
\newtheorem*{terminology}{Terminology}
\newtheorem*{notation}{Notation}
\newtheorem*{convention}{Convention}

\newcommand\Ccancel[2][black]{
    \let\OldcancelColor\CancelColor
    \renewcommand\CancelColor{\color{#1}}
    \cancel{#2}
    \renewcommand\CancelColor{\OldcancelColor}
}

\newcommandtwoopt{\canto}[3][red][0]{
    \let\OldcancelColor\CancelColor
    \renewcommand\CancelColor{\color{#1}}
    \cancelto{#2}{#3}
    \renewcommand\CancelColor{\OldcancelColor}
}




% -------------------------------------------------------------------
% Chapter Headings
% -------------------------------------------------------------------

\setcounter{chapter}{0}

\makeatletter
\renewcommand{\@chapapp}{Lecture}
\makeatother
\definecolor{lightergray}{rgb}{0.9,0.9,0.9}

\usepackage{titlesec}
\titleformat{\section}{\large\bfseries\raggedright}{}{0em}{\colorsection}[\titlerule]
\titleformat{name=\section,numberless}{\large\scshape\bfseries\raggedright}{}{0em}{\colorsectionnonumber}[\titlerule]

\titleformat{\subsection}{\bfseries\raggedright}{}{0em}{\colorsubsection}
\titleformat{name=\subsection,numberless}{\bfseries\raggedright}{}{0em}{\colorsubsectionnonumber}

\newcommand{\colorsection}[1]{%
    \colorbox{lightergray}{\parbox{\dimexpr\textwidth-2\fboxsep}{\thesection\ \ #1}}}
\newcommand{\colorsectionnonumber}[1]{%
    \colorbox{lightergray}{\parbox{\dimexpr\textwidth-2\fboxsep}{#1}}}
    
\newcommand{\colorsubsection}[1]{%
    \colorbox{lightergray}{\parbox{\dimexpr\textwidth-2\fboxsep}{\thesubsection\ #1}}}
\newcommand{\colorsubsectionnonumber}[1]{%
    \colorbox{lightergray}{\parbox{\dimexpr\textwidth-2\fboxsep}{#1}}}
    
\definecolor{gray75}{gray}{0.75}
\newcommand{\hsp}{\hspace{20pt}}
\titleformat{\chapter}[hang]{\Huge\bfseries}{\thechapter\hsp\textcolor{gray75}{|}\hsp}{0pt}{\Huge\bfseries}

\title{Hydrodynamics and elasticity,\\ Lecture notes}
\author{Mikołaj Duch}

\begin{document}

  \maketitle
  \frontmatter

  % -------------------------------------------------------------------
  % Contents
  % -------------------------------------------------------------------

  \tableofcontents

  % -------------------------------------------------------------------
  % Main sections 
  % -------------------------------------------------------------------

  \mainmatter

  \input{sections/01lecture.tex}
  \documentclass[../main.tex]{subfiles}
\begin{document}
  \chapter{Lecture 2}
    \paragraph{Reminder}
    Material derivative 
    \begin{displaymath}
      \dfrac{}{t} = \frac{D}{Dt} = \ptf{}{t} + \vec u \cdot \nabla \cdot
    \end{displaymath}

    Mass conservation implies the continuity equation
    \begin{displaymath}
      \ptf{\rho}{t} + \nabla \cdot ( \rho \vec u ) = 0.
    \end{displaymath}

    \paragraph{New stuff}
    Expanding the above equation we obtain
    \begin{displaymath}
      \underbrace{\ptf{\rho}{t} + \vec u \cdot \nabla \rho}_{\dfrac{\rho}{t}} + \rho \nabla \cdot \vec u = 0,
    \end{displaymath}
    and thus
    \begin{displaymath}
      \frac{1}{\rho} \dfrac{\rho}{t} = - \nabla \cdot \vec u. 
    \end{displaymath}
    If we introduce \fndef{specific volume} $\nu = 1 / \rho$ we obtain
    \begin{displaymath}
      \frac{1}{\nu} \dfrac{\nu}{t} = \nabla \cdot \vec u.
    \end{displaymath}

    We introduced that because we want to study incompressible flow.
    If the flow is incompressible we express it by saying that
    \begin{displaymath}
      \dfrac{\rho}{t} = 0 \quad \tm{or}\quad  \dfrac{\nu}{t} = 0.
    \end{displaymath}
    From the continuity equation incompressibility of the flow implies that
    \begin{displaymath}
      \nabla \cdot \vec u = 0.
    \end{displaymath}
    
    For the incompressible flow the $\vec u$ is divergence-free or solonoidal (\todo I didn't hear well).

    If $\nabla \cdot \vec u = 0$ and $ f(\vec r, t) = f_0 = \tm{const}$ then\footnote{It follows from the continuity equation.}
    \begin{displaymath}
      \forall t \quad f(\vec r, t) = f_0 = \tm{const}.
    \end{displaymath}

    Positive divergence implies expansion, negative implies compression.
    

    \section{Newton's second law (Momentum balance)}
    % \fndef{Material volume}. 
    Consider a closed system (volume $V(t)$) which is comprised of the same fluid particles
    (it flows with a fluid). 

    % \todo Fig0.
    \begin{figure}[h]
      \centering
      \begin{tikzpicture}
        \node[anchor=south west] (image) at (0,0)
        {\includegraphics[width=0.25\textwidth]{./images/01Fig0.png}};
          \begin{scope}[x={(image.south east)}, y={(image.north west)}]
          \node at (0.0, 0.0) {};
          \end{scope}
      \end{tikzpicture}
      % \caption{}
    \end{figure}

    We want to calculate the momentum of such \fndef{material volume}.
    It is obviously an integral
    \begin{displaymath}
      \vec P (t) = \int_{V(t)} \rho(\vec r, t) \vec u(\vec r, t) \d \vec r.
    \end{displaymath}
    That is a linear momentum of the material volume.
    We want to state the Newton second law:
    \begin{displaymath}
      \dfrac{\vec P(t)}{t} = \vec F, 
    \end{displaymath}
    where $\vec F$ is the net force.
    \begin{displaymath}
      \dfrac{\vec P}{t} = \dfrac{}{t} \int_{V(t)} \rho(\vec r, t) \vec u (\vec r, t) \d \vec r = ?.
    \end{displaymath}
    
    Here is a theorem (i.e. fancy name for Leibniz rule):
    
    \begin{theorem}[Raynold's transport theorem]
      \begin{displaymath}
        \dfrac{}{t} \int_{V(t)} \beta (\vec r, t) \d \vec r = 
        \int_{V}\left[ \ptf{\beta}{t} + \nabla \cdot (\beta \vec u) \right] \d \vec r
        = \int_V \left[ \ptf{\beta}{t} + \vec u \cdot \nabla \beta + \beta \nabla \cdot \vec u \right]
        ,
      \end{displaymath}
      where $V$ is a fixed quantity, called control volume (i.e. any volume that coincides with $V(t)$).
    \end{theorem}

    The things that contribute to this change can be interpreted as
    \begin{enumerate}
      \item local change $\ptf{\beta}{t}$,
      \item advection i.e. $\vec u \cdot \nabla \beta$,
      \item changing volume i.e. $\beta \nabla\cdot \vec u$.
    \end{enumerate}

    Applying RTT to the momentum we get
    \begin{displaymath}
      \dfrac{\vec P }{t} = \dfrac{}{t} \int_{V(t)} \rho \vec u \d \vec r 
      = \int_V \left[ \ptf{\rho\vec u }{t} + \nabla \cdot (\rho \underbrace{\vec u \vec u}_{\vec u \otimes \vec u}) \right]
    \end{displaymath}
    
    \paragraph{Homework} Show that if $\beta = \rho b$, then 
    \begin{displaymath}
      \dfrac{}{t} \int_{V(t)} \rho b \d \vec r = \int_{V} \rho \dfrac{b}{t} \d \vec r,
    \end{displaymath}
    using RTT (Raynold's transport theorem) and the continuity equation.
    
    \begin{displaymath}
      \frac{\xi }{ x} = \int_V \d V 
    \end{displaymath}
    
    \begin{displaymath}
      \dfrac{\vec P}{t} = \dfrac{}{t} \int_{V(t)} \rho \vec u \d \vec r 
      = \int_V \rho \dfrac{\vec u}{t} \d \vec r 
      = \vec F,
    \end{displaymath}
    and thus the integral's form of Newton second law
    \begin{displaymath}
      \int_V \rho \dfrac{\vec u}{t} \d \vec r = \vec F.
    \end{displaymath}
    
    \section{Further consequences of RTT}
    % \todo Fig1.

    \begin{figure}[ht]
      \centering
      \begin{tikzpicture}
        \node[anchor=south west] (image) at (0,0)
        {\includegraphics[width=0.25\textwidth]{./images/01Fig1.png}};
          \begin{scope}[x={(image.south east)}, y={(image.north west)}]
          \node at (0.0, 0.0) {};
          \end{scope}
      \end{tikzpicture}
    \end{figure}

    
    \begin{displaymath}
      M = \int_V \rho \d \vec r \qLRa \ptf{M}{t} 
      = \ptf{}{t} \int_V \rho \d \vec r 
      = \int_V \ptf{\rho}{t} \d \vec r 
      = - \int_V \nabla \cdot (\rho \vec u) \d \vec r
      = - \int_{\partial V} \rho \vec u \cdot \hat n \d S .
    \end{displaymath}

    To note: material volume is the volume that flows with the fluid.
    

    Consider a material volume $V(t)$ and its mass given by
    \begin{displaymath}
      M = \int_{V(t)}  \rho \d \vec r.
    \end{displaymath}
    The mass conservation means that
    \begin{displaymath}
      \dfrac{M}{t} = 0.
    \end{displaymath}

    We calculate
    \begin{displaymath}
      \dfrac{M}{t} = \dfrac{}{t} \int_{V(t)} \rho \d \vec r 
      = \int_V \left[ \ptf{\rho}{t} + \nabla \cdot (\rho \vec u) \right]\d \vec r 
      = 0.
    \end{displaymath}
    The above equation states that the mass of the material volume which travels with the flow remains constant.
    
    \paragraph{Force model}
    In the following equation
    \begin{displaymath}
      \int_V \rho \dfrac{\vec u}{t} \d \vec r = \vec F,
    \end{displaymath}
    we do not know what $\vec F$ is and therefore need a model for it.

    % \todo Fig3 TODO
    \begin{figure}[h]
      \centering
      \begin{tikzpicture}
        \node[anchor=south west] (image) at (0,0)
        {\includegraphics[width=0.25\textwidth]{./images/01Fig2.png}};
          \begin{scope}[x={(image.south east)}, y={(image.north west)}]
          \node at (0.0, 0.0) {};
          \end{scope}
      \end{tikzpicture}
    \end{figure}

    Consider that the fluid acts on its surface element $\d a$ (with a normal vector $\hat n$).
    Let $\vec t$ be a force per unit area, and $\d \vec F = \vec t \d a$.
    \begin{displaymath}
      \vec F \os{\tm{model}}{=} \int_{\partial V} \d \vec F 
      = \int_{\partial V}  \vec t \d a
      = - \int_V \nabla p \d \vec r.
    \end{displaymath}
    Assume that $\vec t = - p \hat{n}$, where $p $ is a pressure and $\nabla p $ a \fndef{pressure field}.
    
    \begin{displaymath}
      \int_V \rho \dfrac{\vec u }{t} \d \vec r = - \int_V \nabla p \d \vec r,
    \end{displaymath}
    \begin{displaymath}
      \int_V \left[ \rho \dfrac{\vec u}{t} + \nabla p \right]\d \vec r = 0 
      \qLRa \rho \dfrac{ \vec u }{t} = - \nabla p.
    \end{displaymath}
    We may also write it as 
    \begin{displaymath}
      \rho \left( \ptf{\vec u}{t} + \vec u \cdot \nabla \cdot \vec u  \right) = - \nabla p.
    \end{displaymath}
    This is the \fnvi{Euler model of the ideal fluid} (ideal fluid without dissipation).
    
    \section{Equilibrium}
    Equilibrium is obtained when $\vec u = 0$ and thus $\nabla p = 0$
    (especially if $p = \const$).
    \begin{itemize}
      \item ideal fluid $ \vec t = - p \hat n$
      \item In general $ \vec t = \ten{\Sigma}^T \cdot \vec n$, where $\Sigma$ is a \fndef{Cauchy stress tensor} (second order tensor).
    \end{itemize}
    In general case force can have a form 
    \begin{displaymath}
      \vec F = \int_{\partial V} \vec t \d a = \int_{\partial V} \ten{\Sigma}^T \cdot \hat n \d a 
      = \int_V \nabla \cdot \ten{\Sigma} \d \vec r,
    \end{displaymath}
    with the stress tensor
    \begin{displaymath}
      \ten{\Sigma} = - p \cdot\ten{1} + \ten{\Sigma}'. % TODO that's not all correct. 
    \end{displaymath}


    Newton's second law
    \begin{displaymath}
      \int_V \rho \dfrac{\vec u}{t} \d \vec r = \int\nabla \cdot \ten \Sigma \d \vec r.
    \end{displaymath}

    \begin{displaymath}
      \ten \Sigma = \underbrace{- p \ul{\ul{1}}}_{\tm{ideal term}} + \underbrace{\ul{\ul{\Sigma}}}_{\tm{deviatory part}}
    \end{displaymath}
    Deviatoric part vanishes in equilibrium.

    Ideal fluid model $\ten \Sigma' = 0$, $\ten \Sigma = - p \ul{\ul{1}}$.
    \begin{displaymath}
      \nabla \cdot \Sigma = \nabla \left( - p \ul{\ul{1}} \right) = - \nabla p.
    \end{displaymath}

    \paragraph{Summary}
    Til now we formulated
    
    \begin{itemize}
      \item Continuity equation
        \begin{displaymath}
            \ptf{\rho}{t} + \nabla \cdot ( \rho \vec u ) = 0
        \end{displaymath}
      \item Newton's law
        \begin{displaymath}
          \rho \left( \ptf{\vec u }{t} + \vec u \cdot \nabla \cdot \vec u  \right) = \nabla \cdot \ten \Sigma
      \end{displaymath}
    \item What's next? Angular momentum.
    \end{itemize}

    \section{Angular momentum}
    \begin{figure}[h]
      \centering
      \begin{tikzpicture}
        \node[anchor=south west] (image) at (0,0)
        {\includegraphics[width=0.25\textwidth]{./images/01Fig3.png}};
          \begin{scope}[x={(image.south east)}, y={(image.north west)}]
          \node at (0.0, 0.0) {};
          \end{scope}
      \end{tikzpicture}
    \end{figure}

    Consider a material volume $V(t)$, with density $\rho$ and a point at $\vec r$ moving with a velocity $\vec u$.
    \begin{displaymath}
      \vec L (t) = \int_{V(t)}\vec r \times \rho \vec u \d \vec r
      = \int_{V(t)} \rho (\vec r \times \vec u) \d \vec r
      = \int_{V(t)} \rho \vec l \d \vec r,
    \end{displaymath}
    where $\vec l = \vec r \times \vec u $ is the \fndef{angular momentum per unit mass}.

    The law of the change of the angular momentum
    \begin{displaymath}
      \dfrac{ \vec L }{t} = \vec N,
    \end{displaymath}
    where $\vec N$ is a net torque acting on $V(t)$.
    \begin{displaymath}
      \dfrac{\vec L}{t} = \dfrac{}{t} \int_{V(t)} \rho \vec l \d \vec r \os{\tm{RTT + cont}}{=\joinrel=} \int_V \rho \dfrac{\vec l}{t} \d \vec r.
    \end{displaymath}
    We calculate $\vec N$ by
    \begin{displaymath}
      \vec N = \int_{\partial V} \vec r \times\vec t \d a.
    \end{displaymath}

    \begin{displaymath}
      \int_V \rho \dfrac{\vec l}{t} \d \vec r = \int_{\partial V} \vec r \times \vec t \d a 
      = \int_{\partial V} (\vec r \times \Sigma^T) \cdot \hat n \d a 
    \end{displaymath}
    using divergence theorem
    \begin{equation}
      = \int_V \nabla \cdot \left[ ( \vec r \times \Sigma^T)^T \right] \d \vec r.
      \label{eq:prev}
    \end{equation}

    where the divergence theorem reads
    \begin{displaymath}
      \int_{\partial V} T \cdot \hat n \d a = \int_V \nabla \cdot T^T  \d \vec r.
    \end{displaymath}

    Going back to \ref{eq:prev}
    \begin{displaymath}
       = \int_V \left[ \vec r \times \nabla \cdot \Sigma - 2 \vec \sigma \right] \d \vec r,
    \end{displaymath}
    where $\vec \sigma $ is the axial vector associated with $\Sigma$.
    Thus
    \begin{displaymath}
      \int_V \rho \dfrac{ \vec l}{t} \d \vec r = \int_V \left[ \vec r \times \nabla \cdot \Sigma - 2 \vec \sigma \right] \d \vec r ,
    \end{displaymath}
    and since the volume $V$ can be anything we get
    \begin{equation}
      \rho \dfrac{ \vec l }{t} = \vec r \times \nabla \cdot \Sigma - 2 \vec \sigma.
      \label{eq:2.2}
    \end{equation}

    This is too complicated, we need to simplify it.
    
    \begin{displaymath}
      \rho \dfrac{ \vec l }{t} = \rho \dfrac{\vec r \times \vec u}{t} 
      = \rho \vec r \times \dfrac{\vec u}{t} + \rho \dfrac{ \vec r }{t} \times \vec u
      = \vec r \times \underbrace{\rho \dfrac{\vec u}{t}}_{\nabla \cdot \ten \Sigma} 
      = \vec r \times \nabla \cdot \ten \Sigma.
    \end{displaymath}

    Substituting it to  \ref{eq:2.2} we get
    \begin{displaymath}
      \vec \sigma = 0.
    \end{displaymath}
    Thus the $\Sigma $ is symmetric.

    The stress tensor need not to be symmetric for magnetic fluids (it works for simple fluids).


    \paragraph{,,Complex'' fluids}
    % \begin{figure}
    %   \centering
    %   \begin{tikzpicture}
    %     \node[anchor=south west] (image) at (0,0)
    %     {\includegraphics[width=0.25\textwidth]{./images/01Fig4.png}};
    %       \begin{scope}[x={(image.south east)}, y={(image.north west)}]
    %       \node at (0.0, 0.0) {};
    %       \end{scope}
    %   \end{tikzpicture}
    % \end{figure}

    Consider a magnetic fluid in a bottle, with magnetic dipoles.
    Assume that we apply a magnetic field, so there is a reorientation and an \fndef{internal torque} appears.
    
    \begin{displaymath}
      \vec N = \int_{\partial V} \vec r \times \vec t \d a + \int_{V} \vec b \d \vec r,
    \end{displaymath}
    where $\vec b$ is the internal torque.
    
    
    \section{Energy conservation}
    \begin{figure}
      \centering
      \begin{tikzpicture}
        \node[anchor=south west] (image) at (0,0)
        {\includegraphics[width=0.25\textwidth]{./images/01Fig4.png}};
          \begin{scope}[x={(image.south east)}, y={(image.north west)}]
          \node at (0.0, 0.0) {};
          \end{scope}
      \end{tikzpicture}
    \end{figure}

    Consider a material volume $V$ and a small surface element $\d a$.
    The energy is given by 
    \begin{displaymath}
      E(t) = \int_{V (t)} \rho (\vec r ,t ) e (\vec r, t) \d \vec r ,
    \end{displaymath}
    where $e (\vec r, t)$ is the energy for unit mass.

    The ,,law'' of change
    \begin{displaymath}
      \dfrac{E}{t} = \dfrac{W}{t} + \dfrac{Q}{t},
    \end{displaymath}
    where $W$ is a mechanical work and $Q$ is a heat.

    Using RTT and continuity we get
    \begin{displaymath}
      \dfrac{E}{t} = \dfrac{}{t} \int_{V(t)} \rho e \d \vec r = \int_V \rho \dfrac{e}{t} \d \vec r,
    \end{displaymath}
    
    \begin{displaymath}
      \dfrac{W}{t} = \int_{\partial V } \vec t \cdot \vec u \d a 
      = \int_{\partial V} (\ten \Sigma^T \cdot \vec u ) \cdot \hat n  \d a 
      = \int_V \nabla \cdot ( \ten \Sigma \cdot u) \d \vec r.
    \end{displaymath}

    \begin{displaymath}
      \dfrac{Q}{t} = - \int_{\partial V} \vec q \cdot  \hat n \d a = - \int_V \nabla \cdot \vec q \d \vec r,
    \end{displaymath}
    where $\vec q$ is the heat flow per unit surface per unit time. % TODO upewnić się, że to jest rzeczywiście to 

    Summing up we get
    \begin{equation}
      \rho \dfrac{ e}{t} = \nabla \cdot ( \Sigma \cdot \vec u) - \nabla \cdot \vec q.
      \label{eq:2.3}
    \end{equation}

    Let's introduce the separation 
    \begin{displaymath}
      e = e_0 + \frac{1}{2} \vec u^2.
    \end{displaymath}

    Substituting it into \ref{eq:2.3} we obtain
    \begin{displaymath}
      \rho \dfrac{e_0}{t} = - \rho \dfrac{}{t} \left( \frac{1}{2} \vec u^2 \right) + \nabla \cdot (\Sigma \cdot \vec u ) - \nabla \cdot \vec q.
    \end{displaymath}
    \begin{displaymath}
      \rho \dfrac{}{t} \left( \frac{1}{2} \vec u ^2  \right) 
      = \vec u \cdot \rho \dfrac{\vec u}{t} 
      = \vec u \cdot ( \nabla \cdot \Sigma) 
      = \nabla \cdot ( \Sigma \cdot u) - \Sigma :\nabla \vec u % TODO Double contraction of tensors
    \end{displaymath}
    
    For the internal energy $e_0$:
    \begin{displaymath}
      \rho \dfrac{e_0}{t} = \Sigma : \nabla \vec u - \nabla \cdot \vec q.
    \end{displaymath}
    
    For $\Sigma = - p \ul{\ul{1}} + \Sigma'$
    \begin{displaymath}
      \Sigma : \nabla \vec u = - p ( \nabla \cdot \vec u) + \Sigma' : \nabla \vec u.
    \end{displaymath}
    
    \section{Ideal fluid approximation}

    We assume no deviatoric stress, and no heat flows
    \begin{align*}
      \Sigma' = 0 \\
      \vec q = 0.
    \end{align*}
    For the ideal fluid
    \begin{displaymath}
      \rho \dfrac{e_0}{t} = - p ( \nabla \cdot \vec u).
    \end{displaymath}

    This approximation is a result of assuming that the particles move all together, 
    so there is no viscosity and also no way to conduct a heat.
    
    For a compressible flow
    \begin{itemize}
      \item $\nabla \cdot \vec u > 0 \qLRa \dfrac{e_0}{t} < 0 $ (expansion),
      \item $\nabla \cdot \vec u < 0 \qLRa \dfrac{e_0}{t} > 0 $ (compression),
    \end{itemize}
    and for the incompressible flow there is no way to change internal energy, $\dfrac{e_0}{t} = 0$.
    In ideal fluid there is no dissipation.
    However the real fluids do.

    \section{Entropy balance} 
    \begin{figure}[h]
      \centering
      \begin{tikzpicture}
        \node[anchor=south west] (image) at (0,0)
        {\includegraphics[width=0.25\textwidth]{./images/01Fig5.png}};
          \begin{scope}[x={(image.south east)}, y={(image.north west)}]
          \node at (0.0, 0.0) {};
          \end{scope}
      \end{tikzpicture}
    \end{figure}
    

    Consider a closed system and assume that a heat has been transfered to the system.
    We can move between $S$ and $S + \d S$ by reversible and irreversible paths.
    For the reversible one we have 
    \begin{displaymath}
      \d S  = \frac{\d Q}{T},
    \end{displaymath}
    and for an irreversible proces
    \begin{displaymath}
      \d S \geq \frac{\d Q}{T}.
    \end{displaymath}
    
    \begin{displaymath}
      S(t) = \int_{V(t)} \rho( \vec r , t) s( \vec r, t) \d \vec r,
    \end{displaymath}
    where $s(\vec r, t)$ is entropy per unit volume.
    Entropy balance implies
    \begin{displaymath}
      \dfrac{S}{t} = \frac{\d_e S}{\d t} + \frac{\d_i S}{\d t} 
    \end{displaymath}
    \begin{displaymath}
      \d_e S = \frac{\d Q}{T}.
    \end{displaymath}
    \begin{displaymath}
      \dfrac{S}{t} = \dfrac{}{t} \int_{V(t)} \rho s \d \vec r = \int-\rho \dfrac{s}{t} \d \vec r,
    \end{displaymath}
    \begin{displaymath}
      \dfrac{_e S}{t} = - \int_{\partial V} \frac{ \vec q}{T} \cdot \hat n \d a 
      = - \int_V \nabla \cdot \left( \frac{\vec q}{T} \right) \d \vec r
    \end{displaymath}
    \begin{displaymath}
      \dfrac{_i S}{t} = \int_V \theta \d \vec r,
    \end{displaymath}
    where $\theta$ is the entropy production per unit volume per unit time.
    Thus the \fnte{entropy balance equation} is
    \begin{displaymath}
      \rho \dfrac{s}{t} = - \nabla \cdot \left( \frac{\vec q }{T}  \right) + \theta, \quad \theta \geq 0.
    \end{displaymath}

    For ideal fluid (no internal efects) $\theta = 0$, and the change of entropy
    \begin{displaymath}
      \dfrac{s}{t} = 0.
    \end{displaymath}
\end{document}

  \documentclass[../main.tex]{subfiles}
\begin{document}
  \chapter{Lecture 3}

    Recall
    \begin{displaymath}
      \ptf{\rho}{t} + \nabla \cdot (\rho \vec u) = 0,  
    \end{displaymath}
    \begin{equation}
      \quad \frac{1}{\nu} \dfrac{\nu}{t} = \nabla \cdot \vec u, 
      \quad \nu = \frac{1}{\rho},
      \label{eq:3.1}
    \end{equation}
    \begin{displaymath}
      \rho \dfrac{\vec u }{t} = \rho \left(  \ptf{\vec u}{t} + \vec u \cdot \vec u \right) = \nabla \cdot \Sigma  + \rho g,
    \end{displaymath}
    angular momentum balance
    \begin{displaymath}
      \Sigma^T = \Sigma.
    \end{displaymath}
    energy per unit mass
    \begin{equation}
      \rho \dfrac{e_0}{t} = - p (\nabla \cdot \vec u) + \Sigma' : \nabla \vec u - \nabla \cdot \vec q
      \label{eq:3.2}
    \end{equation}
    entropy per unit mass
    \begin{displaymath}
      \rho \dfrac{s}{t} = - \nabla \cdot \left( \frac{\vec q }{T} \right) + \theta,
    \end{displaymath}
    where $\theta$ is the \fndef{entopy production}. The second law of thermodynamics says that $\theta > 0 $.

    Those are complete system of balance equations and in fact quite general ones.

    \paragraph{Local thermodynamics equilibrium approximation}
    \begin{displaymath}
      \theta = \rho \dfrac{s}{t} + \nabla \cdot \left( \frac{\vec q}{T} \right),
    \end{displaymath}
    \begin{displaymath}
      s = s(e_0, \nu),
    \end{displaymath}
    and the Gibbs relation
    \begin{displaymath}
      \d s = \frac{1}{T}\d e_0 + \frac{p}{T} \d \nu.
    \end{displaymath}
    $\d e_0/\d t$ comes form the flow of a energy and the $\d \nu / \d t$ part comes from continuity equation.
    We can write it as
    \begin{displaymath}
      \rho \dfrac{s}{t} = \frac{1}{T} \underbrace{\rho \dfrac{e_0}{t}}_{\tm{Eq }\ref{eq:3.2}}
      + \frac{p}{T} \underbrace{\frac{1}{\nu} \dfrac{\nu }{t}}_{\tm{Eq } \ref{eq:3.1}},
    \end{displaymath}
    \begin{displaymath}
      \rho \dfrac{s}{t} = \frac{1}{T} \left[ - p (\nabla \cdot \vec u) + \Sigma' : \nabla \vec u - \nabla \cdot \vec q \right]
      + \frac{p}{T} (\nabla \cdot \vec u).
    \end{displaymath}
    \begin{displaymath}
      \nabla \cdot \left( \frac{\vec q }{T} \right) = \frac{1}{T} \nabla \cdot \vec q - \frac{1}{T^2} \vec q \cdot \nabla T,
    \end{displaymath}
    \begin{displaymath}
      \theta =  -\Ccancel[blue]{\frac{p}{T} (\nabla \cdot \vec u)} + \frac{1}{T} \Sigma' : \nabla \vec u -\Ccancel[red]{ \frac{1}{T} \nabla \cdot \vec q}
      + \Ccancel[blue]{\frac{p}{T} (\nabla \cdot \vec u)} + \Ccancel[red]{\frac{1}{T} \nabla \cdot \vec q} - \frac{1}{T^2} \vec q \cdot \nabla T.
    \end{displaymath}
    Thus the entropy production is given by the formula
    \begin{displaymath}
      \theta = - \frac{1}{T^2} \underbrace{\vec q \cdot \nabla T}_{\tm{heat flux}} + \frac{1}{T} \underbrace{\Sigma' : \nabla \vec u}_{\tm{momentum flux}} \geq 0.
    \end{displaymath}
    Imagine that you have a flow and a temperature gradient --- then the fluid will flow from the hotter part to the colder.
    Note that both terms have to be positive, because when one is missing, the other one must be non-negative.

    How does those equations simplify in the ideal fluid model?

    \section{Ideal fluid model}

    Recall that for the ideal fluid we have
    \begin{displaymath}
      \vec q = 0, \quad \Sigma' = 0 \qLRa \theta = 0,
    \end{displaymath}
    and therefore there is no entropy production.

    Hydrodynamics of ideal fluids:
    \begin{enumerate}
      \item $\ptf{\rho}{t} + \nabla \cdot (\rho \vec u) = 0$,
      \item $\rho \dfrac{\vec u}{t} = - \nabla p + \rho \vec g$,
      \item $ \rho \dfrac{e_0}{t} = - p(\nabla \cdot \vec u)$,
      \item $\dfrac{s}{t} = 0$.
    \end{enumerate}
    Since the 2\textsuperscript{nd} equation is a vector one we have $6$ equations. 
    Unfortunately, there are 7 unknowns, which means we are one equation short.
    The one that is missing is the thermodynamical equation of state (for pressure).

    \section{Thermodynamics}
    In the equation
    \begin{displaymath}
      e_0 = e_0(s , \nu)
    \end{displaymath}
    the specific volume as a variable is not that interesting, since it cannot be easily controlled. 
    It would be much better to use a pressure instead of $\nu$. 
    To do that we switch the thermodynamical potential, from energy to enthalpy
    \begin{displaymath}
      h_0 = e_0 + p \nu = e_0 + \frac{p}{\rho}, \quad h_0 = h_0(s, p), 
    \end{displaymath}
    \begin{displaymath}
      \d h_0 = T \d s + \nu \d p = \underbrace{T\d s}_{ = 0} + \frac{\d p}{\rho},
    \end{displaymath}
    if we work in a regime where $s$ is fixed.
    Thus 
    \begin{displaymath}
      \d h_0 = \frac{\d p}{\rho}, \quad h_0 = h_0(p),\quad \ptf{h_0}{p}|_s = \frac{1}{\rho}.
    \end{displaymath}
    Therefore $\rho = \rho(p)$ or $p = p(\rho)$ and that's what we've been missing.
    This equation is called the \fnte{equation of state}.
    
    For an ideal gas
    \begin{displaymath}
      p(\rho) = C \rho ^\gamma, \quad \gamma = \frac{c_p}{c_v}.
    \end{displaymath}

    For an ideal fluid
    \begin{displaymath}
      \rho = \rho_0 = \const.
    \end{displaymath}

    \section{Hydrostatics}
    Now that we have a complete set of equations, we can solve physical problems.
    Assume for now that $\vec u = 0$ which leads to 
    \begin{displaymath}
      \nabla p = \rho \vec g, \quad p = p(\rho).
    \end{displaymath}
    Assume that we work with an ideal gas and know the temperature profile i.e. $T(\tm{height})$.
    
    \begin{figure}[h]
      \centering
      \begin{tikzpicture}
        \draw (0,0) -- (5,0);
        \draw [-stealth] (1, 0) -- (1, 2);
        \node at (4, 1.5) {$ \vec g = - g\vez $};
        \node at (2, 1.5) {$ T(z)$ };
        \node at (0.8, 2) {$z$};
      \end{tikzpicture}
    \end{figure}

    
    \begin{displaymath}
      \dfrac{p}{z} = - \rho g, \quad p(z) = ?,
    \end{displaymath}
    For an ideal gas $p = \rho R T$, thus
    \begin{displaymath}
      \int_{p(z = )}^{p(z)} \frac{\d p'}{p'} = - \int_0^z \frac{g \d z}{RT(z)} 
      \qLRa p(z) = p(0) \exp\left( -\int_0^z \frac{g \d z}{RT(z)} \right),
    \end{displaymath}

    where we've used 
    \begin{enumerate}
      \item $T(z) \ra p(z)$
      \item $\rho(z)$ from the equation $\rho=  \frac{p}{RT}$
    \end{enumerate}

    \paragraph{Sound waves}
    Fluid in equilibrium (neglecting gravity)
    \begin{displaymath}
      \vec u = 0,
    \end{displaymath}
    and $p = p_{og} = \const.$, $\rho = \rho_{og} = \const.$, $s = s_{og} = \const.$.
    A sound wave is just a ,,small'' perturbation of the equilibrium state
    \begin{displaymath}
      \vec u ( \vec r, t) = 0 + \vec u ' (\vec r, t)  \quad (\abs{\vec u'} \ll c)
    \end{displaymath}
    \begin{displaymath}
      p ( \vec r, t) = p_{eq} + p' (\vec r, t)  \quad (p' \ll p_{eq})
    \end{displaymath}
    \begin{displaymath}
      \rho ( \vec r, t) = \rho_{eq} + \rho' (\vec r, t)  \quad (\rho ' \ll \rho_{eq}),
    \end{displaymath}
    \begin{displaymath}
      s( \vec r, t) = s_{eq} = \const.
    \end{displaymath}

    Now we want to find primed variables.
    How? By using the equations of an ideal fluid.
    We have
    \begin{equation}
      \ptf{\rho}{t} + \nabla \cdot (\rho \vec u) = 0,
    \end{equation}
    \begin{equation}
      \rho \left( \ptf{\vec u}{t} + \vec u \cdot \nabla\vec u \right) = - \nabla p,
      \label{eq:3.3}
    \end{equation}
    \begin{equation}
      s = s_{eq} = \const.
    \end{equation}

    The sound velocity squared
    \begin{displaymath}
      \nabla p = \underbrace{\left( \dfrac{p}{\rho} \right)}_{c^2}\nabla \rho = c^2 \nabla \rho.
    \end{displaymath}
    
    Those equations are very hard to solve mathematically, since they are non-linear.
    However, since we just want to solve an easier case, those can be linearly approximated.
    Omitting all second order terms, we obtain %\todo ,,og'' should be changed to ,,eq'' XD.
    \begin{displaymath}
      \ptf{(\rho_{eq} + \rho')}{t} + \nabla \cdot \left[ (\rho_{eq} + \rho') \vec u \right] = 0,
    \end{displaymath}
    and the \fnte{linearized continuity equation}
    \begin{equation}
      \nabla \cdot \vec u' = - \frac{1}{\rho_{eq}} \ptf{\rho'}{t}.
      \label{eq:lincont}
    \end{equation}

    Now we take care of the second equation (\ref{eq:3.3})
    \begin{displaymath}
      (\rho_{eq} + \rho') \left[ \ptf{\vec u'}{t} + \underbrace{\vec u' \cdot \nabla \vec u'}_{\approx 0} \right] = - c^2 \nabla (\rho_{eq} + \rho'),
    \end{displaymath}
    and obtain the \fnte{linearized equation of motion}
    \begin{displaymath}
      \ptf{\vec u'}{t} = - \frac{c^2}{\rho_{eq}} \nabla \rho'.
    \end{displaymath}

    We have the following system
    \begin{align*}
      \nabla \cdot \vec u' = - \frac{1}{\rho_{eq}} \ptf{\rho'}{t}, \\
      \ptf{\vec u'}{t} = - \frac{c^2}{\rho_{eq}} \nabla \rho'.
    \end{align*}
    Differentiating the first equation with respect to time and using divergence operator on the second one we get
    \begin{displaymath}
      \dptf{(\rho')}{t} = c^2 \nabla^2 \rho'.
    \end{displaymath}

    So the speed of sound
    \begin{displaymath}
      c^2 = \left( \ptf{p}{\rho} \right)_s.
    \end{displaymath}
    For $s = \const$, $p(\rho) = C \rho^\gamma$, it can be approximated as
    \begin{displaymath}
      \left( \ptf{p}{\rho} \right)_s = \gamma \frac{p}{\rho} \approx \gamma\frac{p_{eq}}{\rho_{eq}}.
    \end{displaymath}
    \begin{displaymath}
      c^2 \approx \gamma \frac{p_{eq}}{\rho_{eq}} = \gamma R T_{eq}.
    \end{displaymath}
    

    \paragraph{1-D wave equation}
    \begin{equation}
      \dptf{(\rho')}{t} = c^2 \dptf{(\rho')}{x}, \quad \rho'(x, t) = ?.
      \label{eq:1Dwave}
    \end{equation}
    To solve it we write $\rho'$ in a Fourier representation
    \begin{displaymath}
      \rho'(x,t) = \int\d k \int \d \omega \rho'(k, \omega) e^{ik x} e^{-i\omega t}.
    \end{displaymath}
    Substituting the above into \ref{eq:1Dwave} we get
    \begin{displaymath}
      (- i \omega) ^2 \rho'(k, \omega) = c^2 (i k)^2 \rho'(k,\omega),
    \end{displaymath}
    \begin{displaymath}
      \omega^2 = c^2 k^2 \qLRa \omega = \pm c k.
    \end{displaymath}
    The above equation is called the \fndef{dispersion relation}.
    Using it we get
    \begin{displaymath}
      \exp\left[ i (k x - \omega t) \right] = \exp\left[ i k (x - c t)\right].
    \end{displaymath}

    Finally
    \begin{displaymath}
      \rho'(x, t) = \int \d k \rho_1'(x) \exp\left( i k (x - c t) \right) + \int \d k \rho_2' \exp(ik ( x + ct)),
    \end{displaymath}
    which are two families of perturbations travelling in the opposite directions.

    \paragraph{3D case}

    Using Fourier representation
    \begin{displaymath}
      \rho'(\vec r, t) = \int \d \vec k \int \d \omega \rho'(\vec k, \omega) \exp\left[ i( \vec k \cdot \vec r - \omega t )\right],
    \end{displaymath}
    \begin{displaymath}
      \dptf{(\rho')}{t} = c^2 \nabla^2 \rho',
    \end{displaymath}
    \begin{displaymath}
      (- i\omega)^2 \rho'(\vec k, \omega) - c^2 (i\vec k)^2 \rho'(\vec k, \omega) \qLRa \omega = \pm c \abs{k},
    \end{displaymath}
    \begin{displaymath}
      \exp\left[ i(\vec k \cdot \vec r - \omega t) \right] = \exp\left[ i k ( \hat{k} \cdot \vec r - c t \right].
    \end{displaymath}
    Note that $c$ doesn't depend on $\omega$. 
    If it would, we will call such wave \fndef{dispersive}.
    Nondispersive waves travel without changing shape.

    We know $\rho'(\vec r, t)$. What about $p'(\vec r, t)$ and $\vec u'(\vec r, t)$?

    By substituting into
    \begin{displaymath}
      \d p = \left( \frac{p}{\rho} \right)_s \d \rho, 
    \end{displaymath}
    relations $p'(\vec r, t) = c^2 \rho'(\vec r, t)$, we can calculate $p'(\vec r, t)$.

    To obtain velocity $\vec u$ we note that
    \begin{displaymath}
      \ptf{\vec u'}{t}  = - \frac{c^2}{\rho_{eq}} \nabla \rho'
    \end{displaymath}
    and, by using Fourier transform, we obtain
    \begin{displaymath}
      \vec u' (\vec k, \omega) = \frac{c^2}{\rho_{eq}\omega}\vec k \rho'(\vec k, \omega) 
      = \pm\frac{c^2}{\rho_{eq} ck } \vec k \rho'(\vec k, \omega) 
    \end{displaymath}
    \begin{displaymath}
      \vec u' = \pm  \frac{c}{\rho_{eq}} \rho' \hat k.
    \end{displaymath}
    Sound waves are longitudinal (particles move in the same way the waves propagates).

    \section{Aerodynamics}

    Imagine that there is a wing profile and consider a wind tunel configuration.

    \begin{figure}[h]
      \centering
      \begin{tikzpicture}
        \node[anchor=south west] (image) at (0,0)
        {\includegraphics[width=0.25\textwidth]{./images/02Fig1.png}};
          \begin{scope}[x={(image.south east)}, y={(image.north west)}]
          % \node at (0.0, 0.0) {};
          \draw [-stealth] (0.5, 0.45) -- (0.6, 0.85);
          \draw [dashed, red, -stealth] (0.5, 0.45) -- (0.5, 0.85);
          \draw [dashed, green, -stealth] (0.5, 0.45) -- (0.6, 0.45);
          \node at (0.4, 0.8) {lift};
          \node at (0.6, 0.3) {drag};
          \end{scope}
      \end{tikzpicture}
      \label{fig:}
    \end{figure}

    Question: what is the force that the flow exerts on the object?
    The force can be decomposed into two parts: the lift and the drag.
    \begin{displaymath}
      \vec F = -\int_S p \hat{n} \d s, \quad p(\vec r, t) = ?
    \end{displaymath}

    To obtain $p$, $\vec u$ and $\vec F$ we have to solve Euler's equations
    \begin{align*}
      \rho\left( \ptf{\vec u}{t} + \vec u \cdot \nabla \vec u \right) = - \nabla p + \rho \vec g\\
      \ptf{p}{t} + \rho \cdot (\rho \vec u) = 0.
    \end{align*}
    Those are most often almost impossible to solve.
    We should try to simplify the mathematical problem here.

    First we can try choosing a different variable.
    We change $\vec u$ into \fndef{vorticity} $\vec \xi = \nabla \times \vec u$.
    
    \begin{figure}
      \centering
      \begin{tikzpicture}
        \node[anchor=south west] (image) at (0,0)
        {\includegraphics[width=0.65\textwidth]{./images/02Fig2.png}};
          \begin{scope}[x={(image.south east)}, y={(image.north west)}]
          \node at (0.0, 0.0) {};
          \end{scope}
      \end{tikzpicture}
      \label{fig:}
    \end{figure}

    Euler's equations
    \begin{displaymath}
      \ptf{\vec u}{t}+ \vec u \cdot \nabla \cdot \vec u= - \frac{1}{\rho} \nabla p + \vec g
    \end{displaymath}
    \begin{enumerate}
      \item $ \vec g = - \nabla \chi$
      \item 
        \begin{displaymath}
          \vec u \cdot \nabla \cdot \vec u = \nabla \left( \frac{\vec u^2}{2} \right) + (\nabla \times \vec u) \times \vec u 
          = \nabla \left( \frac{\vec u^2}{2} \right) + \vec \xi \times \vec u
        \end{displaymath}
      \item $ \frac{1}{\rho} \nabla p = \nabla h_0$
    \end{enumerate}
    \begin{equation}
      \ptf{\vec u}{t} + \nabla \times \vec u = - \nabla h_0 - \nabla \left( \frac{\vec u^2}{2} \right) - \nabla \chi
      = - \nabla \left( h_0 + \frac{1}{2} \vec u^2 \chi \right) =: - \nabla  \mathfrak{h},
      \label{eq:3.4}
    \end{equation}
    where
    \begin{displaymath}
      \mathfrak{h} = h_0 + \frac{ \vec u^2}{2} + \chi .
    \end{displaymath}

    Then Equation \ref{eq:3.4} reads
    \begin{displaymath}
      \ptf{\vec u}{t} + \vec \xi \times \vec u = - \nabla \mf{h}.
    \end{displaymath}
    By taking the rotation of both sides 
    \begin{displaymath}
      \ptf{\overbrace{\nabla \times \vec u}^{\xi}}{t} + \nabla \times (\vec \xi \times \vec u) = 0,
    \end{displaymath}

    we obtain
    \begin{displaymath}
      \nabla \times (\vec \xi \times \vec u) = ( \vec u \cdot \nabla) \vec \xi - (\vec \xi \cdot \nabla) \vec u 
      + \xi (\nabla \cdot \vec u) - \vec u (\nabla \cdot \vec \xi)=
    \end{displaymath}
    if the flow is incompressible
    \begin{displaymath}
      \ptf{\vec \xi}{t} + \vec u \cdot \nabla \vec \xi = \vec \xi \cdot \nabla \cdot \vec u
    \end{displaymath}
    \begin{displaymath}
      \dfrac{\vec \xi}{t} = \vec \xi \cdot \nabla \vec \xi.
    \end{displaymath}

    If $\nabla \vec u = 0$ the $\xi $ is an invariant of a motion.
\end{document}

  
  \chapter{Lecture 4}
  \section{Recall what we already know}
  For the ideal fluid the stress tensor consists only of $p$ and an identity tensor.
  However, real fluids are not ideal. 
  Newton's second law for real fluid
  \begin{displaymath}
    \rho \dfrac{\vec u}{t} = - \nabla p + \vec f, \quad \rho \dfrac{\vec u}{t} = \div \ten \Sigma + \vec f.
  \end{displaymath}

  It would be perfect if we knew the \fnte{heat equation} i.e. $p = p(\rho, T)$, 
  but for now we only have the equation of state $p(\rho)$.

  Other equations that we have
  \begin{enumerate}
    \item momentum equation
    \item mass conservation
    \item equation of state
    \item continuity equation
  \end{enumerate}
  
  For an incomprehensible fluid
  \begin{displaymath}
    \ptf{\rho}{t} + \nabla \cdot(\rho \vec u) = 0 \qLRa \rho (\nabla \cdot \vec u) = 0.
  \end{displaymath}
  

  % TODO name of this equation.
  \begin{displaymath}
    \ptf{\vec u}{t} + (\nabla \times \vec u) \times \vec u = - \nabla (\frac{\vec u^2}{2} + \varphi + \psi),
  \end{displaymath}
  where $\vec f = - \nabla \varphi$, $\psi = \frac{p}{\rho}$.
  By introducting vorticity $\xi = \nabla \times \vec u$ we can write it as
  \begin{displaymath}
    \ptf{\vec u}{t} + \xi \times \vec u = - \nabla (\frac{\vec u^2}{2} + \varphi + \psi)
  \end{displaymath}

  
  $\frac{1}{2}\xi$ represents the averege angular velocity of this initially $\perp$ segments in the fluid.
  Consider a $2$D fluid $\vec \xi = \xi \vez$, $\xi = \ptf{u_y}{x} - \ptf{u_x}{y}$, and look on two
  small, perpendicular segments. We will consider a difference between their components velocities in the $y$ direction.
  \begin{figure}[h]
    \centering
    \begin{tikzpicture}[scale = 2]
      \draw (0, 0) -- (0, 1);
      \draw (0, 0) -- (1, 0);
      \node at (-0.2, 1) {C};
      \node at (-0.2, -0.2) {A};
      \node at (1, -0.2) {B};
      \node at (-0.2, 0.5) {$\delta_y$};
      \node at (0.5, -0.2) {$\delta_x$};
    \end{tikzpicture}
    \label{fig:}
  \end{figure}
  % Let us calculate the angular velocity of
  \begin{displaymath}
    u_y(B) - u_y(A) = u_y(x + \delta, y)- u_y(x, y) 
    = \Ccancel[blue]{u_y(x, y)} + \ptf{u_y}{x} \delta x - \Ccancel[blue]{u_y(x, y)}
    = \ptf{u}{x} \delta x.
  \end{displaymath}
  This is an instantaneous  angular velocity of $AB$ around the $\perp$ axis through $A$.
  Computing the same for rotation along $C$ axis we get
  \begin{displaymath}
    u_x(C) - u_x(A) = \ptf{u_x}{y} \delta_y.
  \end{displaymath}
  So the vorticity at a point informs us about how much will rotate two, initially close, points. % TODO
  Vorticity is a measure of rotation, but rather nonintuitive.
  
  Example. Take rigid body motion $\vec u = \vec \Omega \times \vec r$.
  \begin{displaymath}
    \vec \xi = \nabla \times \vec u , \quad \xi_i 
    = \epsilon_{ijk} \ptf{u_k}{x^j}
    = \epsilon_{ijk} \epsilon_{klm} \Omega_l \overbrace{\ptf{r_m}{x^j}}^{\delta_{jm}}
    = \epsilon_{ijk} \epsilon_{klj} \Omega_l 
    = - \left( \delta_{il} - 3 \delta_{il} \right) \Omega_l = 2 \Omega_i.
  \end{displaymath}
  Thus, for this particular flow
  \begin{displaymath}
    \vec \xi = 2 \vec \Omega.
  \end{displaymath}
  Note that the left side refers to local rotation, and right refers to global rotation.

  \paragraph{Nonintuitive case}
  Consider bathtub vortex.
  It can be represented as
  \begin{displaymath}
    \vec u = \frac{k}{r} \hat{e}_\theta.
  \end{displaymath}
  
  Trick
  \begin{displaymath}
    \nabla \times \vec u = \frac{1}{r} \begin{bmatrix}
      \ver & r \hat{e} _\theta & \hat{e}_z\\
      \ptf{}{r} & \ptf{}{\theta} & \ptf{}{z} \\
      u_r & r u_\theta & u_z
    \end{bmatrix} = 0.
  \end{displaymath}
  It means that the amount of global and local rotation perfectly cancel out, and the local vorticity meter 
  shows nothing.

  \section{Irrotational flows}
  
  A flow with $\vec \xi = 0$ is called \fndef{irrotational}.
  \begin{displaymath}
    \left.\begin{matrix}
      \nabla \times \vec u = 0 \\
      \nabla \times \nabla \chi = 0 
    \end{matrix}\right\} \quad \vec u =  \nabla \chi,
  \end{displaymath}
  which is called \fndef{potential flow}.
  $\chi$ can be defind as
  \begin{displaymath}
    \chi(\vec r) = \int_{r_0}^{\vec r} \vec u \d \vec r', 
    \quad \chi(\vec r + \d \vec r) - \chi(\vec r) = \nabla \chi \cdot \d \vec r = \vec u \cdot \d \vec r,
  \end{displaymath}
  Note. $\chi$ is determined uniquely for simply connected domains (with no holes).

  From the stokes theorem
  \begin{displaymath}
    \oint_{1-2} \vec u \cdot \d \vec r = \int\d S( \nabla \times \vec u) = 0, 
  \end{displaymath}
  since $\nabla \times \vec u$.
  
  \section{Bernoulli theorem}

  Euler's equation
  \begin{displaymath}
    \rho\dfrac{\vec u}{t} = - \nabla p + \vec f.
  \end{displaymath}
  We make the following assumptions
  \begin{enumerate}
    \item potential form $\vec f = - \nabla \varphi$
    \item incomprehensible flow $\rho = \const$
  \end{enumerate}
  Note that
  \begin{displaymath}
    \dfrac{\vec u}{t} = \ptf{\vec u}{t} + (\vec u \cdot \nabla ) \vec u 
    = \ptf{\vec u}{t} + (\nabla \times \vec u) \times \vec u + \nabla \left( \frac{1}{2} \vec u^2 \right),
  \end{displaymath}
  thus we can rewrite
  \begin{displaymath}
    \ptf{\vec u}{t} + 
    % \Ccancel[red]{\vec \xi \times \vec u} 
    % \renewcommand\CancelColor{\color{red}}
    % \cancelto{0}{\vec \xi \times \vec u} 
    \canto{\vec \xi \times \vec u}
    = - \nabla \left( \frac{\vec u^2}{2} + \varphi + \psi\right) , \quad \psi = \frac{p}{\rho},
  \end{displaymath}
  since flow is irrotational.
  For a barotropic fluid
  \begin{displaymath}
    \frac{1}{\rho} \nabla p = \nabla \psi, \quad \psi = \int\frac{\d p'}{\rho(p')}.
  \end{displaymath}
  Now we can write
  \begin{displaymath}
    \ptf{\vec u}{t} = \ptf{\nabla \chi}{t} = \nabla \ptf{\chi}{t}.
  \end{displaymath}
  
  Gathering everything on a one side we get
  \begin{displaymath}
    \nabla \left( \ptf{\chi}{t} + \frac{\vec u^2}{2} + \psi + \varphi \right) = 0,
  \end{displaymath}
  and also
  \begin{displaymath}
    \ptf{\chi}{t} + \frac{\vec u^2}{2} + \psi + \varphi  = C(t),
  \end{displaymath}
  which is \fnte{Cauchy first integral of the Euler equation for an irrotational form}.

  This can be further simplified by setting
  \begin{displaymath}
    \chi' = \chi + \int C(t) \d t, \quad \nabla \chi' = \nabla \chi
  \end{displaymath}
  then
  \begin{displaymath}
    \ptf{\chi'}{t} + \frac{\vec u^2}{2} + \varphi + \psi = 0.
  \end{displaymath}
  This works everywhere in the fluid and is called \fnte{Bernoulli's theorem}.
  If the flow if \fndef{steady} then $\ptf{\chi'}{t} = 0$ and
  \begin{displaymath}
    \frac{\vec u^2}{2} + \varphi + \psi = \const.
  \end{displaymath}
  This is \fnte{Bernoulli's theorem for steady irrotational flow}.

  \section{Bernoulli's theorem for rotational forms}
  
  Lamb's form 
  \begin{displaymath}
    \ptf{\vec u}{t} + \vec \xi \times \vec u = - \nabla \left( \frac{\vec u^2}{2} + \varphi + \psi \right),
  \end{displaymath}
  but there is no velocity potential.
  Consider steady flow and perform scalar multiplication by $\vec u$.
  We get 
  \begin{displaymath}
    0 = \vec u \cdot (\vec \xi \times \vec u) 
    = - \vec u \cdot\nabla \left(\frac{\vec u^2}{2} + \varphi + \psi\right)
    = \vec u \cdot \nabla H, \quad H = \frac{1}{2} \vec u^2 + \varphi + \psi.
  \end{displaymath}
  
  The quantity $H$ is constatn along streamlines!

  Evangelista Torricelli in $1664$ asked the following problem.
  A barrel of wine has a little spout at the bottom. 
  If we remove a plag we see a stream of fluid.
  How long will it take for the barrel to drain?

  Assuming that we deal with an incompressible fluid, $\psi = \frac{p}{\rho}$, and $\varphi = g z$
  Choose a streamline and two points on it $A$, $B$.
  By using second Bernoulli equation we can calculate a velocity of an outgoing fluid.
  It will be equal
  \begin{displaymath}
    H_A = gh + \frac{p_0}{\rho} + \frac{1}{2} \cdot 0 ^2 ,
  \end{displaymath}
  \begin{displaymath}
    H_B = 0 + \frac{p_0}{\rho} + \frac{1}{2} \vec u^2.
  \end{displaymath}
  Comparing those two we obtain
  \begin{displaymath}
    \frac{p_0}{\rho} + gh = \frac{p_0}{\rho} + \frac{1}{2} \vec u^2 \qiff V = \sqrt{2gh}.
  \end{displaymath}

  For a bottle with diameter $1$ m and height $2$ m, and $a = 5$ cm, $V = 6.3$ $\frac{\tm{m}}{\tm{s}}$.
  
  \section{Vorticity equation}
  \begin{displaymath}
    \ptf{\vec u}{t} + \vec \xi \times \vec u = - \nabla H
  \end{displaymath}
  \begin{displaymath}
    \ptf{\vec \xi}{t} + \nabla \times (\vec \xi \times \vec u) = - \nabla \times \nabla H = 0
  \end{displaymath}
  \begin{displaymath}
    \nabla \times(\vec \xi \times\vec u )  = (\vec u \cdot \nabla) \vec \xi + 
    \Ccancel[red]{\vec \xi (\nabla \cdot \vec u)} - \Ccancel[blue]{(\nabla \cdot \vec \xi) \vec u} - (\vec \xi \cdot \nabla ) \vec u.
  \end{displaymath}
  thus
  \begin{displaymath}
    \ptf{\vec \xi}{t} + (\vec u\cdot \nabla ) \vec \xi = (\vec \xi \cdot \nabla)\vec u 
    \qLRa \dfrac{\vec \xi}{t} = (\vec \xi \cdot \nabla) \vec u.
  \end{displaymath}

  For a 2D flow
  \begin{displaymath}
    \vec u = \begin{pmatrix}
      u_x\\
      u_y\\
      0
    \end{pmatrix}, \quad 
    \vec \xi = \begin{pmatrix}
      0\\
      0\\
      \xi
    \end{pmatrix}, \quad (\vec \xi \cdot \nabla) \vec u = 0, 
  \end{displaymath}
  thus
  \begin{displaymath}
    \dfrac{\vec \xi}{t} = 0.
  \end{displaymath}

  \begin{enumerate}
    \item For a $2$D flow vorticity of a fluid element is conserved.
    \item In a 3D, if at any time $t_0$ $\vec \xi =0$ then for $t> t_0$ it will remain $0$.
      (Persistance of irrotational flows, Cauchy-Lagrange theorem)
    \item Consider a steady flow. Then 
      \begin{displaymath}
        (\vec u \cdot \nabla) \vec \xi = 0.
      \end{displaymath}
      
  \end{enumerate}

  \section{Circulation}
  An ideal fluid is sometimes called inviscid.
  Consider force field $\vec g = - \nabla g$ and consider a material curve made of fluid elements.

  Define a \fndef{circulation}  along a curve $c(t)$, which is equal to 
  \begin{displaymath}
    \Gamma(t) = \oint_{c(t)} \vec u \cdot \d \vec r.
  \end{displaymath}
  Kelvin Circualtion theorem: $\Gamma(t) = \const$

  \begin{proof}
    
    We calculate the change on the circulation while the $c(t)$ is changing.
    Let's look at it in an Euler picture.
    \begin{displaymath}
      \delta \Gamma(c(t),t) = \Gamma(c(t + \delta t), t + \delta t) - \Gamma(c(t), t) 
      = \oint_{c(t + \delta t)} \vec u( r', t + \delta t) \d \vec r' 
      - \oint_{c(t + \delta t)} \vec u( r') \d \vec r' 
    \end{displaymath}
    \begin{displaymath}
      \delta \Gamma(t) = \oint_{c(t)} \vec u( \vec r + \vec u(\vec r, t) \delta t, t + \delta t) \cdot ( \d \vec r + ( \d \vec r \cdot \nabla) \vec u \delta t
      - \oint_{c(t)}\vec u(\vec r, t) \d \vec r = 
    \end{displaymath}
    expand to first order in $\delta t$
    \begin{displaymath}
      = \oint_{c(t)} \left\{ \ptf{\vec u}{t} + (\vec u(\vec r, t)\cdot \nabla ) \vec u\delta t) \cdot \d \vec r + \vec u (\vec r, t) \cdot (\d \vec r \cdot \nabla ) \vec u \delta t \right\}
      =\int_{c(t)} (- \nabla \frac{p}{\rho} - \nabla \varphi + \frac{1}{2} \nabla \vec u^2\d \vec r  = \int\nabla \dots\tm{\todo}.
    \end{displaymath}
  \end{proof}

  \chapter{Lecture 5}
  \section{Magnus effect}
  \todo Fig0
  Consider a flow around a circular cylinder of a radius $a$ with
  \begin{displaymath}
    u_r = 0, \quad u_\theta = - 2 U \sin \theta.
  \end{displaymath}
  Points where velocity is equal to $0$ are called \fndef{stagnation points}.
  We denote them by $S_1, S_2$.
  Recall that there is no force acting on the circle! (D'Alambert paradox).

  We will do something artificial to obtain the force.
  Assme that the cylinder rotates.
  A \fndef{circulation} is a flow which flow lines are circles.
  A \fndef{free vortex} is 
  \begin{displaymath}
    \vec u_{\tm{vortex}} = \frac{\Gamma}{2 \pi r} \hat{e}_\theta,
  \end{displaymath}
  where $\Gamma$ is just 
  \begin{displaymath}
    \Gamma = \oint_C \vec u \cdot \d \vec r,
  \end{displaymath}
  i.e. a circulation associated with the flow.
  We have to satisfy the boundary conditions and they happen to be satisfied when we superpose original flow with 
  the artificial one.
  \begin{displaymath}
    \nabla \times \vec u_{\tm{vortex}} = \vec 0,
  \end{displaymath}
  \begin{displaymath}
    \vec u_{\tm{vortex}} = \nabla \phi_{\tm{vortex}}.
  \end{displaymath}
  Thus
  \begin{displaymath}
    \frac{1}{r} \ptf{\Phi_{\tm{vortex}}}{\theta} = u_\theta = \frac{\Gamma}{2 \pi r}, \quad
    \ptf{\Phi_{\tm{vortex}}}{\theta} = \frac{\Gamma}{2\pi}, \quad \Phi_{\tm{vortex}} = \frac{\Gamma}{2\pi}\theta.
  \end{displaymath}
  \begin{displaymath}
    \Phi = U \left( r+ \frac{a^2}{r}  \right) \cos \theta + \frac{\Gamma}{2\pi}\theta.
  \end{displaymath}
  Velocity field $\vec u = \nabla \Phi$
  \begin{displaymath}
    u_r = \ptf{\Phi}{r} = U \left( 1 - \frac{a^2}{r^2} \right) \cos \theta,
  \end{displaymath}
  \begin{displaymath}
    u_\theta = \frac{1}{r} \ptf{\Phi}{\theta} = - U \left( 1 + \frac{a^2}{r^2} \right) \sin \theta + \frac{\Gamma}{2\pi r}.
  \end{displaymath}
  For $r = a$ we have $u_r = 0$, $u_\theta = - 2 U \sin \theta + \Gamma/(2 \pi a)$.
  The stagnation points move downwords.

  \todo Fig1
  
  Position of stagnation poins are at points that satisfy
  \begin{displaymath}
    u_\theta = 0 = 2 U \sin \theta - \frac{\Gamma}{2 \pi a} \qLLa \sin \theta = \frac{\Gamma}{4\pi U a} = \frac{y_s}{a}.
  \end{displaymath}
  Thus $y_s = - a $ implies $\Gamma = - 4 \pi U a$.
  If we increase $U$ even more the saturation points may not be on the surface of the circle.
  
  \todo Fig2 
  \todo Fig3
  
  Lets calculate the force 
  \begin{displaymath}
    \vec F = - \int_{\partial V} p(r = a, \theta) \hat n d S, 
  \end{displaymath}
  and Bernoulli
  \begin{displaymath}
    \frac{p}{\rho} + \frac{\vec u ^2 }{2} = \const. \qLRa p = \const. - \frac{\rho \vec u^2}{2}.
  \end{displaymath}
  Therefore, 
  \begin{displaymath}
    p(r = 0, \theta) = \const - \frac{1}{2}\left[ 2 U \sin \theta - \frac{\Gamma}{2 \pi a} \right]^2,
  \end{displaymath}
  \begin{displaymath}
    \hat n = \cos \theta \vex + \sin \theta \vey, \quad \d S = aL \d \theta ,
  \end{displaymath}
  where $L$ is the length of the cylinder.
  Due to the symmetry we expect that $ \vec F \sim \vey$.
  \begin{displaymath}
    F_x = \frac{1}{2} \rho \int_0^{2 \pi} \left[ 2 U\sin \theta - \frac{\Gamma}{2\pi a} \right]^2 \cos \theta aL\d \theta = 0,
    \quad 
    F_y =  \frac{1}{2} \rho \int_0^{2 \pi} \left[ 2 U\sin \theta - \frac{\Gamma}{2\pi a} \right]^2 \sin \theta aL\d \theta = - \rho U \Gamma.
  \end{displaymath}
  It is called the \fndef{Magnus effect}.
  
  Kutta-Joukowski law states that, for any configuration of the boundary with the rotating ,,artificial'' flow, the force is given by
  \begin{displaymath}
    F_x = 0, \quad F_y = - p U \Gamma.
  \end{displaymath}
  
  
  \section{Flow around the plane wing}
  We will do so by using the conformal mapping form a cylinder to a wing.
  But first, we need some mathematical tools
  We assume that our problem is two-dimensional, $\nabla \times \vec u = 0$, $ \nabla \cdot \vec u = 0$.
  Thus we have the potential $\Phi$ such that $\vec u = \nabla \Phi$, $\nabla^2 \Phi = 0$.
  Also, we have $\Psi$ such that $\vec u =  \nabla \times \vec \psi \vez$, $\nabla^2 \Psi = 0$.
  Thus, for $\vec u = (u, v)$ 
  \begin{displaymath}
    u = \ptf{\Phi}{x} = \ptf{\psi}{y}, \quad v = \ptf{\Phi}{y}  = - \ptf{\Psi}{x}.
  \end{displaymath}

  Consider a complex plane with $z = x + i y$ and an analytic funciton 
  \begin{displaymath}
    w(z) = \Phi(x,y) + i \Psi(x,y),
  \end{displaymath}
  called complex velocity potential.
  Thus, complex velocity
  \begin{displaymath}
    \dfrac{w}{z} = \lim_{\Delta z \ra 0} \frac{w(z + \Delta z) - w(z)}{\Delta z}.
  \end{displaymath}

  \begin{displaymath}
    \dfrac{w}{z} = u - i v, \quad \abs{\dfrac{w}{z}}^2 = u^2 + v^2.
  \end{displaymath}

  \paragraph{Example.} Consider an uniform flow $\vec u = U \vex = (U, 0)$.
  Thus
  \begin{displaymath}
    \ptf{\Phi}{x} = U, \quad \Phi = U x,
  \end{displaymath}
  \begin{displaymath}
    \ptf{\Psi}{y} = U, \quad \Psi = Uy, 
  \end{displaymath}
  \begin{displaymath}
    w(z) = U x + i U y.
  \end{displaymath}

  \paragraph{Free vortex}
  \begin{displaymath}
    \vec u = \frac{\Gamma}{2 \pi r} \hat e_\theta, \quad w(z) = ?,
  \end{displaymath}
  \begin{displaymath}
    \frac{1}{r} \ptf{\Phi}{\theta} = \frac{\Gamma}{2 \pi r} \qLRa \Phi = \frac{\Gamma}{2\pi}\theta,
  \end{displaymath}
  \begin{displaymath}
    - \ptf{\Psi}{r} = \frac{\Gamma}{2\pi r} \qLRa \Psi = - \frac{\Gamma}{2\pi} \log r,
  \end{displaymath}
  \begin{displaymath}
    w(z) = \Phi + i \Psi = \frac{\Gamma}{2 \pi}\theta - i \frac{\Gamma}{2\pi} \log r = \frac{\Gamma}{2 \pi} ( \theta - i \log r).
  \end{displaymath}
  Substituting $z = r e^{i \theta}$ we obtain 
  \begin{displaymath}
    w(z) = - i \frac{\Gamma}{2 \pi}\log z.
  \end{displaymath}
  
  \paragraph{Cylinder without circluation}
  \begin{displaymath}
    \phi= U \left(r + \frac{a^2}{r}\right) \cos \theta,
  \end{displaymath}
  \begin{displaymath}
    \Psi = U\left( r - \frac{a^2}{r}\right) \sin \theta,
  \end{displaymath}
  \begin{displaymath}
    w(z) = U\left( z + \frac{a^2}{z} \right).
  \end{displaymath}
  
  \paragraph{Cylinder with circulation}
  We just superpose the free vortex one with the cylinder without circulation to obtain
  \begin{displaymath}
    w(z) = U \left( z + \frac{a^2}{z} \right) - i \frac{\Gamma}{2 \pi }\log z
  \end{displaymath}
  
  \paragraph{Plane wing}

  Assume that we know the solution for the problem when the wing is just a cylinder
  and also that we know the mapping from cylindrical problem into the physical one.
  Assume that $z = z(z_1)$ and also we know the inverse $z_1 = z_1(z)$.
  Then, our solution is given by
  \begin{displaymath}
    w(z) = w_1[z_1(z)].
  \end{displaymath}
  
  \paragraph{Joukowski transformation}
  Consider
  \begin{displaymath}
    Z = z_1 + \frac{a^2}{z_1}, \quad a \in \RR
  \end{displaymath}
  and
  \begin{displaymath}
    z_1 = \rho e^{i \phi}, \quad \rho \neq a.
  \end{displaymath}
  Then,
  \begin{displaymath}
    Z = x + iy 
    = \rho e^{i \varphi} + \frac{a^2}{\rho} e^{-i \varphi} 
    = \left( \rho + \frac{a^2}{\rho} \right) \cos \varphi + i\left( \rho - \frac{a^2}{\rho} \right) \sin \varphi.
  \end{displaymath}
  
  \todo FigNext (XD)

  We want to calculate a force on a plate that forms an angle $\alpha$.
  Consider a cylinder with
  \begin{displaymath}
    w_1(z_1) = U\left(z + \frac{a^2}{z_1}\right).
  \end{displaymath}
  To obtain a plate we first need to rotate the cylinder by $\alpha$, thus multiply by $z_1e^{i \alpha}$.
  We get $z_2 = z_1 e^{i\alpha}$. Next we squeeze using the Joukowski transformation
  \begin{displaymath}
    z = z_2 + \frac{a^2}{z2}.
  \end{displaymath}
  Using the formula for the derivative of a composite function we obtain
  \begin{displaymath}
    \dfrac{w}{z} = \dfrac{w_1}{z_1} \dfrac{z_1}{z_2}\dfrac{z_2}{z} = \dfrac{w_1}{z_1} e^{-i\alpha} \left[ 1 - \frac{a^2}{z_2^2} \right]^2.
  \end{displaymath}
  To obtain the force we use Kutta-Joukowski theorem.
  That requires calculating the circulation first.
  But we used only conformal mapping, thus the circulation is the same as in the case of the cylinder.
  Thus the force is 0 again XD.

  Note that we have a singularity for $z_1 = a$. 
  It is called a \fndef{trailing edge}.
  To solve this problem we say that $\d w_1/ \d z_1 = 0$ and the trailing edge is a stagnation point.
  
  We introduce an artificial circulation to make $z = a$ a stagnation point.
  It is called a Kutta condition.
  The circulation that we have to put is equal to 
  \begin{displaymath}
    \alpha = - \theta_s, \quad \sin \theta_s = \frac{\Gamma}{4 \pi U a}.
  \end{displaymath}

  Now the circulation is non-zero and thus
  \begin{displaymath}
    \Gamma = -4\pi U a \sin \alpha, \quad F = 4 \pi U a \rho \sin \theta.
  \end{displaymath}
  
  \todo loads of figs
  
  A \fndef{stall} is a phenomenon when $\alpha$ is big enough that the force starts to decrease.
  In case of real fluids we also have the \fndef{turbulent wake}.


  
  

  

  
  
  
  
  
  
  
  
  
  
  
  

  
  
  
  

  
  
  
  
  
  
  
  
  

  

  
\end{document}
