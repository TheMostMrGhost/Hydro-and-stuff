\documentclass[../main.tex]{subfiles}
\begin{document}
  \chapter{Lecture 5}
  \section{Magnus effect}
  \todo Fig0
  Consider a flow around a circular cylinder of a radius $a$ with
  \begin{displaymath}
    u_r = 0, \quad u_\theta = - 2 U \sin \theta.
  \end{displaymath}
  Points where velocity is equal to $0$ are called \fndef{stagnation points}.
  We denote them by $S_1, S_2$.
  Recall that there is no force acting on the circle! (D'Alambert paradox).

  We will do something artificial to obtain the force.
  Assme that the cylinder rotates.
  A \fndef{circulation} is a flow which flow lines are circles.
  A \fndef{free vortex} is 
  \begin{displaymath}
    \vec u_{\tm{vortex}} = \frac{\Gamma}{2 \pi r} \hat{e}_\theta,
  \end{displaymath}
  where $\Gamma$ is just 
  \begin{displaymath}
    \Gamma = \oint_C \vec u \cdot \d \vec r,
  \end{displaymath}
  i.e. a circulation associated with the flow.
  We have to satisfy the boundary conditions and they happen to be satisfied when we superpose original flow with 
  the artificial one.
  \begin{displaymath}
    \nabla \times \vec u_{\tm{vortex}} = \vec 0,
  \end{displaymath}
  \begin{displaymath}
    \vec u_{\tm{vortex}} = \nabla \phi_{\tm{vortex}}.
  \end{displaymath}
  Thus
  \begin{displaymath}
    \frac{1}{r} \ptf{\Phi_{\tm{vortex}}}{\theta} = u_\theta = \frac{\Gamma}{2 \pi r}, \quad
    \ptf{\Phi_{\tm{vortex}}}{\theta} = \frac{\Gamma}{2\pi}, \quad \Phi_{\tm{vortex}} = \frac{\Gamma}{2\pi}\theta.
  \end{displaymath}
  \begin{displaymath}
    \Phi = U \left( r+ \frac{a^2}{r}  \right) \cos \theta + \frac{\Gamma}{2\pi}\theta.
  \end{displaymath}
  Velocity field $\vec u = \nabla \Phi$
  \begin{displaymath}
    u_r = \ptf{\Phi}{r} = U \left( 1 - \frac{a^2}{r^2} \right) \cos \theta,
  \end{displaymath}
  \begin{displaymath}
    u_\theta = \frac{1}{r} \ptf{\Phi}{\theta} = - U \left( 1 + \frac{a^2}{r^2} \right) \sin \theta + \frac{\Gamma}{2\pi r}.
  \end{displaymath}
  For $r = a$ we have $u_r = 0$, $u_\theta = - 2 U \sin \theta + \Gamma/(2 \pi a)$.
  The stagnation points move downwords.

  \todo Fig1
  
  Stagnation poins are at points that satisfy
  \begin{displaymath}
    u_\theta = 0 = 2 U \sin \theta - \frac{\Gamma}{2 \pi a} \qLRa \sin \theta = \frac{\Gamma}{4\pi U a} = \frac{y_s}{a}.
  \end{displaymath}
  Thus $y_s = - a $ implies $\Gamma = - 4 \pi U a$.
  If we increase $U$ even more the saturation points may not be on the surface of the circle.
  
  \todo Fig2 
  \todo Fig3
  
  Lets calculate the force 
  \begin{displaymath}
    \vec F = - \int_{\partial V} p(r = a, \theta) \hat n d S, 
  \end{displaymath}
  and Bernoulli
  \begin{displaymath}
    \frac{p}{\rho} + \frac{\vec u ^2 }{2} = \const. \qLRa p = \const. - \frac{\rho \vec u^2}{2}.
  \end{displaymath}
  Therefore, 
  \begin{displaymath}
    p(r = 0, \theta) = \const - \frac{1}{2}\left[ 2 U \sin \theta - \frac{\Gamma}{2 \pi a} \right]^2,
  \end{displaymath}
  \begin{displaymath}
    \hat n = \cos \theta \vex + \sin \theta \vey, \quad \d S = aL \d \theta ,
  \end{displaymath}
  where $L$ is the length of the cylinder.
  Due to the symmetry we expect that $ \vec F \sim \vey$.
  \begin{displaymath}
    F_x = \frac{1}{2} \rho \int_0^{2 \pi} \left[ 2 U\sin \theta - \frac{\Gamma}{2\pi a} \right]^2 \cos \theta aL\d \theta = 0,
  \end{displaymath}
  \begin{displaymath}
    F_y =  \frac{1}{2} \rho \int_0^{2 \pi} \left[ 2 U\sin \theta - \frac{\Gamma}{2\pi a} \right]^2 \sin \theta aL\d \theta = - \rho U \Gamma.
  \end{displaymath}
  
  It is called the \fndef{Magnus effect}.
  
  Kutta-Joukowski law states that, for any configuration of the boundary with the rotating ,,artificial'' flow, the force is given by
  \begin{displaymath}
    F_x = 0, \quad F_y = - \rho U \Gamma.
  \end{displaymath}
  
  
  \section{Flow around the plane wing}
  We will do so by using the conformal mapping form a cylinder to a wing.
  But first, we need some mathematical tools
  We assume that our problem is two-dimensional, $\nabla \times \vec u = 0$, $ \nabla \cdot \vec u = 0$.
  Thus we have the potential $\Phi$ such that $\vec u = \nabla \Phi$, $\nabla^2 \Phi = 0$.
  Also, we have $\Psi$ such that $\vec u =  \nabla \times \vec \psi \vez$, $\nabla^2 \Psi = 0$.
  Thus, for $\vec u = (u, v)$ 
  \begin{displaymath}
    u = \ptf{\Phi}{x} = \ptf{\psi}{y}, \quad v = \ptf{\Phi}{y}  = - \ptf{\Psi}{x}.
  \end{displaymath}

  Consider a complex plane with $z = x + i y$ and an analytic funciton 
  \begin{displaymath}
    w(z) = \Phi(x,y) + i \Psi(x,y),
  \end{displaymath}
  called complex velocity potential.
  Thus, complex velocity
  \begin{displaymath}
    \dfrac{w}{z} = \lim_{\Delta z \ra 0} \frac{w(z + \Delta z) - w(z)}{\Delta z}.
  \end{displaymath}

  \begin{displaymath}
    \dfrac{w}{z} = u - i v, \quad \abs{\dfrac{w}{z}}^2 = u^2 + v^2.
  \end{displaymath}

  \paragraph{Example.} Consider a uniform flow $\vec u = U \vex = (U, 0)$.
  Thus
  \begin{displaymath}
    \ptf{\Phi}{x} = U, \quad \Phi = U x,
  \end{displaymath}
  \begin{displaymath}
    \ptf{\Psi}{y} = U, \quad \Psi = Uy, 
  \end{displaymath}
  \begin{displaymath}
    w(z) = U x + i U y.
  \end{displaymath}

  \paragraph{Free vortex}
  \begin{displaymath}
    \vec u = \frac{\Gamma}{2 \pi r} \hat e_\theta, \quad w(z) = ?,
  \end{displaymath}
  \begin{displaymath}
    \frac{1}{r} \ptf{\Phi}{\theta} = \frac{\Gamma}{2 \pi r} \qLRa \Phi = \frac{\Gamma}{2\pi}\theta,
  \end{displaymath}
  \begin{displaymath}
    - \ptf{\Psi}{r} = \frac{\Gamma}{2\pi r} \qLRa \Psi = - \frac{\Gamma}{2\pi} \log r,
  \end{displaymath}
  \begin{displaymath}
    w(z) = \Phi + i \Psi = \frac{\Gamma}{2 \pi}\theta - i \frac{\Gamma}{2\pi} \log r = \frac{\Gamma}{2 \pi} ( \theta - i \log r).
  \end{displaymath}
  Substituting $z = r e^{i \theta}$ we obtain 
  \begin{displaymath}
    w(z) = - i \frac{\Gamma}{2 \pi}\log z.
  \end{displaymath}
  
  \paragraph{Cylinder without circluation}
  \begin{displaymath}
    \phi= U \left(r + \frac{a^2}{r}\right) \cos \theta,
  \end{displaymath}
  \begin{displaymath}
    \Psi = U\left( r - \frac{a^2}{r}\right) \sin \theta,
  \end{displaymath}
  \begin{displaymath}
    w(z) = U\left( z + \frac{a^2}{z} \right).
  \end{displaymath}
  
  \paragraph{Cylinder with circulation}
  We just superpose the free vortex one with the cylinder without circulation to obtain
  \begin{displaymath}
    w(z) = U \left( z + \frac{a^2}{z} \right) - i \frac{\Gamma}{2 \pi }\log z
  \end{displaymath}
  
  \paragraph{Plane wing}

  Assume that we know the solution for the problem when the wing is just a cylinder
  and also that we know the mapping from cylindrical problem into the physical one.
  Assume that $z = z(z_1)$ and also we know the inverse $z_1 = z_1(z)$.
  Then, our solution is given by
  \begin{displaymath}
    w(z) = w_1[z_1(z)].
  \end{displaymath}
  
  \paragraph{Joukowski transformation}
  Consider
  \begin{displaymath}
    Z = z_1 + \frac{a^2}{z_1}, \quad a \in \RR
  \end{displaymath}
  and
  \begin{displaymath}
    z_1 = \rho e^{i \phi}, \quad \rho \neq a.
  \end{displaymath}
  Then,
  \begin{displaymath}
    Z = x + iy 
    = \rho e^{i \varphi} + \frac{a^2}{\rho} e^{-i \varphi} 
    = \left( \rho + \frac{a^2}{\rho} \right) \cos \varphi + i\left( \rho - \frac{a^2}{\rho} \right) \sin \varphi.
  \end{displaymath}
  
  \todo FigNext (XD)

  We want to calculate a force on a plate that forms an angle $\alpha$.
  Consider a cylinder with
  \begin{displaymath}
    w_1(z_1) = U\left(z + \frac{a^2}{z_1}\right).
  \end{displaymath}
  To obtain a plate we first need to rotate the cylinder by $\alpha$, thus multiply by $z_1e^{i \alpha}$.
  We get $z_2 = z_1 e^{i\alpha}$. Next we squeeze using the Joukowski transformation
  \begin{displaymath}
    z = z_2 + \frac{a^2}{z2}.
  \end{displaymath}
  Using the formula for the derivative of a composite function we obtain
  \begin{displaymath}
    \dfrac{w}{z} = \dfrac{w_1}{z_1} \dfrac{z_1}{z_2}\dfrac{z_2}{z} = \dfrac{w_1}{z_1} e^{-i\alpha} \left[ 1 - \frac{a^2}{z_2^2} \right]^2.
  \end{displaymath}
  To obtain the force we use Kutta-Joukowski theorem.
  That requires calculating the circulation first.
  But we used only conformal mapping, thus the circulation is the same as in the case of the cylinder.
  Thus the force is 0 again XD.

  Note that we have a singularity for $z_1 = a$. 
  It is called a \fndef{trailing edge}.
  To solve this problem we say that $\d w_1/ \d z_1 = 0$ and the trailing edge is a stagnation point.
  
  We introduce an artificial circulation to make $z = a$ a stagnation point.
  It is called a Kutta condition.
  The circulation that we have to put is equal to 
  \begin{displaymath}
    \alpha = - \theta_s, \quad \sin \theta_s = \frac{\Gamma}{4 \pi U a}.
  \end{displaymath}

  Now the circulation is non-zero and thus
  \begin{displaymath}
    \Gamma = -4\pi U a \sin \alpha, \quad F = 4 \pi U a \rho \sin \theta.
  \end{displaymath}
  
  \todo loads of figs
  
  A \fndef{stall} is a phenomenon when $\alpha$ is big enough that the force starts to decrease.
  In case of real fluids we also have the \fndef{turbulent wake}.
  
\end{document}
