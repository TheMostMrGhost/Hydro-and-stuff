\documentclass[../main.tex]{subfiles}
\begin{document}
  \chapter{Lecture 6}

  Demonstration of a difference between water and corn syroup.
  The latter has a high viscosity.

  A viscous fluid starts flowing under shear. % shearing motion
  \fndef{Shear stress} means naprężenia ścinające.
  A \fndef{Newtonian fluid} resist its changing of form proportionally to the rate of deformation.

  \paragraph{The rate of deformation}
  We consider two points, one at $r_0$ and the other at $r_0 + \d r_0$, and consider their motion to 
  $\vec r(r_0 + \d r_0, t)$ and $\vec r(r_0, t)$ respectively.
  We have
  \begin{displaymath}
    \d \vec r= \vec r(\vec r_0 + \d \vec r_0, t) - \vec r ( \vec r_0, t).
  \end{displaymath}
  How fast does $\d \vec r$ change while moving along the trajectory of motion?
  To calculate the velocity we need a material derivative of $\vec r$.
  We have (in the Lagrange picture)
  \begin{align*}
    \frac{D}{Dt} \d \vec r &= \ptf{}{t} \vec r (\vec r_0 + \d \vec r, t) - \ptf{}{t}\vec r(\vec r_0, t)\\
    &= \vec v(\vec r_0 + \d \vec r_0, t) - \vec v(\vec r_0, t)
  \end{align*}
  switching to Euler
  \begin{align*}
    \frac{D}{Dt} \d \vec r &= \vec v(\vec r + \d \vec r, t) - \vec v(\vec r, t)\\
    &= (\nabla \vec v) \cdot \d \vec r.
  \end{align*}
  The object $\nabla \vec v$ is a velocity gradient, with coordinates given by
  \begin{displaymath}
    \frac{D}{Dt} (\d r)_i = (\nabla \cdot \vec v)_i^j \d r_j, 
  \end{displaymath}
  thus
  \begin{displaymath}
    \frac{D}{Dt} \d \vec r = (\nabla \vec v) \cdot \d \vec r.
  \end{displaymath}
  We build \fndef{rate of deformation tensor} as
  \begin{displaymath}
    \hat D = (\nabla \vec v)^S = \frac{1}{2} \left[ (\nabla \vec v) + (\nabla \vec v)^T \right].
  \end{displaymath}
  We want to connect the stress in the fluid to the deformation.
  Recall
  \begin{displaymath}
    \ten \Sigma  = - p \ten 1.
  \end{displaymath}

  We assume that stress is linear to deformation (in a Newtonian fluid) and the fluid is isotropic.
  We would like to know what is the most general, linear relationship between them.
  It is given by
  \begin{displaymath}
    \Sigma_{ij} = C_{ijkl} D_{kl}.
  \end{displaymath}
  Fun fact: there is only one isotropic, symmetric tensor of order $4$ in dimension $3$
  \begin{displaymath}
    C_{ijkl} = \alpha \delta_{ij} \delta_{lk} + \beta \delta_{ik}\delta_{jl} + \gamma \delta_{il} \delta_{jk}.
  \end{displaymath}
  Thus
  \begin{displaymath}
    \Sigma_{ij} = \alpha \delta_{ij} D_{lk} + \beta D_{ij} + \gamma D_{ji} = \alpha \delta_{ij} D_{ll} + (\beta + \gamma) D_{ij}.
  \end{displaymath}
  
  Let us denote $\lambda = \alpha _{ij}$, $2 \mu = \beta + \gamma$. 
  Thus
  \begin{displaymath}
    \ten \Sigma' = \lambda (\tr( \ten D)) \ten 1 + 2\mu \ten D,
  \end{displaymath}
  \begin{displaymath}
    \tr \ten \Sigma' = (3 \lambda + 2 \mu) \tr(D) = 3 \xi \tr (D),
  \end{displaymath}
  where $\xi = \lambda + \frac{2}{3} \mu$ and is called a \fndef{bulk viscosity} and $\mu$ is 
  called a \fndef{shear viscosity}.

  The equations of motion (momentum eq.)
  \begin{displaymath}
    \rho \frac{D\vec v}{D t} = \div \ten \Sigma + \vec f.
  \end{displaymath}
  We know something about the relation between $\Sigma$ and $\frac{D \vec v}{Dt}$ so we can calculate the closed form.
  \begin{displaymath}
    (\div (- p \vec 1))_j = \ptf{}{x^i} (- p \delta_{ij}) = -\ptf{p}{x^j}.
  \end{displaymath}

  \paragraph{Barotropic fluids}
  We want to calculate $\div \ten \Sigma'$.
  \begin{displaymath}
    (\div \ten \Sigma')_i = \ptf{\Sigma'_{ij}}{x^j} = 2 \mu \ptf{D_{ij}}{x^j} + \lambda \delta_{ij} \ptf{D_{kk}}{x^j}
  \end{displaymath}
  inserting
  \begin{displaymath}
    D_{ij} = \frac{1}{2} \left( \ptf{v_j}{x^i} + \ptf{v_i}{x^j} \right) 
  \end{displaymath}
  into the above equation we obtain
  \begin{displaymath}
    (\div \ten \Sigma')_i = \mu \left( \frac{\partial^2 v_j}{\partial x^j \partial x^i} 
    + \frac{\partial^2 v_k}{\partial x^j \partial x^k}\right) + \lambda \frac{\partial^2 v_k}{\partial x^j \partial x^k} \delta_{ij}.
  \end{displaymath}

  Thus we have
  \begin{displaymath}
    (\div \ten \Sigma')_i = \mu \nabla^2 v_i + (\lambda + \mu) \frac{\partial^2 v_j}{\partial x^j \partial x^i} 
    = \mu \nabla^2 + (\lambda + \mu) [\nabla (\nabla \cdot \vec v)]_i
  \end{displaymath}

  Using the equations
  \begin{displaymath}
    \rho \left( \ptf{\vec v}{t} + (\vec v \cdot \nabla) \vec v \right) 
    = - \nabla p + \mu \nabla^2 \vec v + (\xi + \frac{1}{3} \mu) \nabla (\nabla \cdot \vec v) + \vec f
  \end{displaymath}
  \begin{displaymath}
    \ptf{\rho}{t} + \nabla \cdot (\rho \vec v) = 0, \quad p = p(\rho).
  \end{displaymath}

  \paragraph{Intuition}
  Assume a unidirectional flow along $x$, varying along $y$ with
  \begin{displaymath}
    \vec v = [v_x(y), 0, 0].
  \end{displaymath}
  Lot of things simplifies, for example
  \begin{displaymath}
    \nabla \cdot \vec v = 0.
  \end{displaymath}
  
  Let's calculate the deformation tensor
  \begin{displaymath}
    \ten D = (\nabla \vec v)^S = \frac{1}{2} \begin{pmatrix}
      0 & \ptf{v_x}{y} & 0 \\
       \ptf{v_x}{y} & 0 & 0 \\
       0 & 0 & 0
    \end{pmatrix}, \quad \ten \Sigma' = \lambda (\tr \ten D) \ten 1 + 2 \mu \ten D 
    = \mu \begin{pmatrix}
      0 & \ptf{v_x}{y} & 0\\
      \ptf{v_x}{y} & 0 & 0\\
      0 & 0 & 0
    \end{pmatrix}.
  \end{displaymath}
  What is the interpretation of such results?
  \begin{displaymath}
    \vec f= \ten \Sigma \cdot \vec n, \quad f_i = \Sigma_{ij} \vec n_j, 
  \end{displaymath}
  with $\vec n$ as the normal vector, we get
  \begin{displaymath}
    \vec f = \vec f_n + \vec f_s,
  \end{displaymath}
  where the first is the normal component and the second is shear component.
  Choose $\vec n = \vey$. We have
  \begin{displaymath}
    \vec f = \ten{ \Sigma} \hat e_j = \mu \begin{pmatrix}
      0 & \ptf{v_x}{y} & 0\\
      \ptf{v_x}{y} & 0 & 0\\
      0 & 0 & 0
    \end{pmatrix} \cdot \begin{pmatrix}
      0 \\ 
      1 \\
      0
    \end{pmatrix} = \begin{pmatrix}
      \mu \ptf{v_x}{y}\\
      0\\
      0
    \end{pmatrix}.
  \end{displaymath}
  This means that there is purely shear stress, with the relation between rate of change of velocity 
  between subsequent fluid layer given by
  \begin{displaymath}
    \Sigma_{xy} = \ptf{v_x}{y}.
  \end{displaymath}
  
  
  We point out that $[\mu] =\tm{Pa} \cdot S$.
  $\mu $ is a \fndef{dynamic viscosity}.
  We can write $N \cdot S$ as
  \begin{displaymath}
    \frac{D \vec v}{Dt} =\frac{1}{\rho} \nabla p + \nu \nabla^2 \vec v + \frac{\vec f}{\rho},
  \end{displaymath}
  where $\nu = \mu/\rho$ and is called a \fndef{kinematic viscosity}.
  
    \begin{table}
      \label{tab:}
      \begin{center}
        \begin{tabular}[c]{c|c|c|c}
          \hline
          Example & $\rho$ kg/m$^3$ & $\mu$ / Pa$\cdot$s & $\nu$ m$^3$/s \\
          \hline
          hydrogen & 0.084 & $8.8\cdot 10^{-5}$ & $10^{-4}$ \\
          \hline
          air & 1.18 & $1.8\cdot 10^{-5}$ & $1.5 \cdot 10^{-5}$ \\
          \hline
          water & 1000 & $10^{-3}$ & $10^{-6}$ \\
          \hline
        \end{tabular}
      \end{center}
    \end{table}

    \section{Boundary conditions}
    \paragraph{1)}
    On the surface of a solid body $\vec v = 0$, $\vec v_n = 0$, $\vec v_t = 0$.
    Those are called stick (or not slim) boundary conditions.
    Otherwise infinite gradients.

    \paragraph{2)} On a free surface $\vec v_n = 0$. 
    A tangential velocity is not necesserily zero, but is not determined a priori.
    The normal stresses have to be continuous, thus $\ten \Sigma \cdot \vec n$ are continuous.

    \section{Bulk viscosity}
    For an ideal fluid
    \begin{align*}
      \ten \Sigma &= - p \ten 1 + \ten \Sigma^T \\
      &= - p \ten 1 + (\xi - \frac{2}{3} \mu) (\nabla \cdot \vec v) \ten 1 + \mu [(\nabla \vec v) + (\nabla \vec v)^T] \\
      &= - p \ten 1 + \xi (\nabla \cdot \vec v) \ten 1 + \mu [(\nabla \vec v) + (\nabla \vec v) ^T - \frac{2}{3} (\nabla \cdot \vec v) \ten 1] \\
      &= 0,
    \end{align*}
    thus $\ten \Sigma$ is symmetric and stressless tensor.

    For a viscous fluid
    \begin{displaymath}
      p^* = - \frac{1}{3} \tr \ten \Sigma = p - \xi(\nabla \cdot \vec v).
    \end{displaymath}
    The part $p$ is a pressure and $\xi (\nabla \cdot \vec v) $ is a dynamic pressure.
    The second part is important only when the fluid is compressible (for example in sound waves).

    \section{Reynolds number}
    Assume incompressible fluid.
    We introduce one number to rule them all --- the \fndef{Reynolds number}.
    We performed an experiment \todo Fig.
    \fndef{Laminar flow} --- continuous, paralell streamlines.
    After increasing the flow speed he observed irregularities.
    At some point it starts to go crazy, for example completely mix with a fluid.

    Consider a steady flow of a viscous fluid.
    \begin{align*}
      \rho(\vec v \cdot \nabla) \vec v = - \nabla p + \mu \nabla^2 \mu\\
      \nabla \cdot \vec v = 0.
    \end{align*}

    Consider a flow with characteristic length $L$ and velocity $U$.
    Those scale as
    \begin{displaymath}
      \rho( \vec v \cdot \nabla) \vec v \sim \rho U \cdot \frac{U}{L}
    \end{displaymath}
    \begin{displaymath}
      \mu \nabla^2 \vec v \sim \mu \frac{U}{L^2}.
    \end{displaymath}
    Thus the Reynolds number 
    \begin{displaymath}
      R_e = \frac{\tm{inertia}}{\tm{viscosity}} 
      = \frac{\rho \frac{U^2}{L}}{\mu \frac{U}{L^2}} 
      = \frac{\rho UL}{\mu}
      = \frac{UL}{\nu}.
    \end{displaymath}
    We have the following possibilities
    \begin{itemize}
      \item $R_e \ll 1$ inertia $\ll$ viscosity, laminar flow (Stokes flow),
      \item $R_e \approx 1$ can still be laminar,
      \item $R_e \gg 1$ inertia dominated flow --- turbulent flow.
    \end{itemize}
    The exact number for which there occurs transition from laminar to turbulent depends on the problem.
    Typically it is around 1000.
    For water $10^{-6}$, for a bacteria $R_e = 10^{-6}$, for honey on toast $R_e = 10^{-3}$, 
    for a swimmer in a pool
\end{document}
